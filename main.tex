\documentclass[12pt,a4paper]{article}

% --------------------
% Pacotes básicos
% --------------------
\usepackage[utf8]{inputenc}   % Codificação de caracteres
\usepackage[T1]{fontenc}      % Acentuação correta
\usepackage[brazil]{babel}    % Idioma português
\usepackage{geometry}         % Controle de margens
\usepackage{hyperref}         % Links clicáveis no PDF
\usepackage{setspace}         % Espaçamento
\usepackage{enumitem}         % Personalização de listas
\usepackage{graphicx}         % Para inserir imagens
\usepackage{titlesec}         % Formatação de títulos

% --------------------
% Configurações
% --------------------
\geometry{a4paper, margin=2.5cm}
\setstretch{1.3} % Espaçamento melhorado
\hypersetup{
    colorlinks=true,
    linkcolor=black,
    urlcolor=blue,
    citecolor=blue,
    bookmarks=true,
    bookmarksopen=true,
    pdfstartview=FitH
}

% Configurações de listas
\setlist{itemsep=0.5em, parsep=0pt, topsep=0.5em}
\setlist[enumerate,1]{itemsep=0.8em}
\setlist[itemize,1]{itemsep=0.5em}

% Configurações de seções
\usepackage{titlesec}
\titlespacing*{\section}{0pt}{3ex plus 1ex minus .2ex}{2.3ex plus .2ex}
\titlespacing*{\subsection}{0pt}{2.5ex plus 1ex minus .2ex}{1.5ex plus .2ex}
\titlespacing*{\subsubsection}{0pt}{2ex plus 1ex minus .2ex}{1ex plus .2ex}

% --------------------
% Início do documento
% --------------------
\begin{document}

\title{
    \textbf{Projeto de Engenharia de Software} \\[0.5em]
    \large Sistema de Gestão dos Restaurantes Universitários da UFF \\[0.3em]
    \normalsize SG-RU/UFF
}
\author{
    José Enrique Viana de Oliveira \\
    Rafael Rocha Damasceno Ferreira \\
    Eduardo Rottschaefer Oliveira \\
    Alessandro Felipe Ferreira Filho \\ 
    Felipe Cabral Liporage \\
    Allan Vignoli dos Santos \\
    João Victor Machado Sperle \\
    João Vitor Pereira Rodrigues\\
    Vinícius Barreto Pinheiro\\
}
\date{\today}

\maketitle
\thispagestyle{empty}

\newpage
\tableofcontents
\newpage

% --------------------
% Seção: Escopo
% --------------------
\section{Escopo}

\subsection{Nome provisório do projeto}
Sistema de Gestão dos Restaurantes Universitários da UFF (SG-RU/UFF).

\subsection{Descrição}
Desenvolver e operar um sistema integrado para os Restaurantes Universitários (RUs) da UFF em Niterói que gerencie:
\begin{enumerate}
    \item Fila virtual de acesso;
    \item Estoque de insumos e refeições prontas;
    \item Logística entre unidades;
    \item Cardápio e informações nutricionais.
\end{enumerate}
O sistema terá interfaces para perfis administrativos e será acessível aos usuários finais (discentes, docentes e servidores) através do aplicativo IdUFF.

% --------------------
% Seção: Requisitos Funcionais
% --------------------
\newpage
\section{Requisitos Funcionais}

\subsection{Requisitos Gerais}
\begin{enumerate}[label=\textbf{RF-GEN-\arabic*}, leftmargin=*, align=left]
    \item \textbf{Autenticação via IdUFF (SSO).} O sistema deve permitir login do usuário utilizando o SSO institucional (IdUFF), garantindo sessão segura e expiração configurável. % (equivale ao rf1.1)

    \item \textbf{Interfaces por perfil.} O sistema deve exibir interfaces distintas para: (a) administradores/operadores do RU e (b) usuários finais (discentes, docentes e servidores) que utilizam o RU. % (equivale ao rf1.2)

    \item \textbf{Níveis de acesso (RBAC).} O sistema deve suportar perfis e papéis com permissões diferenciadas (por exemplo: Administrador, Operador de Cozinha, Logística, Nutrição/Qualidade, Usuário Final), restringindo ações e telas conforme o papel. % (equivale ao rf1.3)

    
\end{enumerate}

\subsection{Requisitos da Fila para Usuários}
\begin{enumerate}[label=\textbf{RF-FIL-U-\arabic*}, leftmargin=*, align=left]
    \item O sistema deve possuir um sistema de fila virtual associado ao ingresso no RU.
    \item O sistema deve permitir que o usuário entre na fila virtual.
    \item O sistema deve permitir que o usuário saia da fila virtual.
    \item O sistema deve exibir ao usuário sua posição atual na fila.
    \item O sistema deve notificar o usuário sobre a evolução de sua posição na fila.
    \item O sistema deve permitir que o usuário personalize suas preferências de notificação (ex.: push, frequência).
    \item O sistema deve permitir diferentes níveis de prioridade ao entrar na fila, conforme políticas institucionais (idosos, PCDs, bolsistas, etc.).
    \item O sistema deve exibir ao usuário uma estimativa de tempo de espera (ETA).
    \item O sistema deve exibir avisos ao usuário em caso de imprevistos operacionais no RU (ex.: atrasos, falta de energia).
    \item O sistema deve permitir que o usuário confirme ou desista de sua posição na fila quando solicitado.
    \item O sistema deve oferecer a possibilidade de o usuário retroceder voluntariamente posições na fila.
    \item O sistema deve permitir que os usuários enviem feedbacks sobre a experiência na fila virtual.
\end{enumerate}

\subsection{Requisitos da Fila para Administradores}
\begin{enumerate}[label=\textbf{RF-FIL-A-\arabic*}, leftmargin=*, align=left]
    \item O sistema deve exibir em tempo real o número de pessoas na fila, o tempo médio de espera e os horários de pico.
    \item O sistema deve receber informações das catracas físicas presentes no local para registrar pessoas entrando e saindo do RU.
    \item O sistema deve permitir que os administradores monitorem quantas pessoas estão simultaneamente no refeitório.
    \item O sistema deve controlar a capacidade máxima de pessoas simultaneamente no refeitório, bloqueando novas entradas quando necessário.
    \item O sistema deve gerar relatórios históricos de fluxo de pessoas e de refeições servidas.
  
    \item O sistema deve permitir a gestão e configuração das regras de prioridade de entrada (idosos, gestantes, PCDs, bolsistas, etc.).
\end{enumerate}


\subsection{Requisitos de Estoque}
\begin{enumerate}[label=\textbf{RF-EST-\arabic*}, leftmargin=*, align=left]
    \item O sistema deve armazenar a quantidade de ingredientes em estoque.
    \item O sistema deve armazenar a quantidade de alimentos prontos para consumo em estoque.
    \item O sistema deve permitir que o administrador edite a quantidade de ingredientes em estoque.
    \item O sistema deve permitir que o administrador edite a quantidade de alimentos prontos para consumo em estoque.
    \item O sistema deve permitir que o administrador consulte a quantidade de cada ingrediente não preparado.
    \item O sistema deve permitir que o administrador consulte a quantidade de cada alimento pronto para consumo.
    \item O sistema deve permitir que o administrador informe quando um certo item, em um determinado RU, acabou.
    \item O sistema deve permitir que o administrador faça um pedido de envio de mais comida.
    \item O sistema deve notificar o responsável pelo envio de comida sobre os pedidos recebidos.
    \item O sistema deve permitir que o administrador responsável pelo envio receba feedback do pedido realizado.
    \item O sistema deve possibilitar que os administradores monitorem as informações da viagem de envio (tempo estimado, item pedido, veículo de transporte, motorista responsável).
    \item O sistema deve permitir que o administrador registre relatórios de consumo de ingredientes e refeições em diferentes períodos (diário — almoço/janta, semanal, mensal).
    \item O sistema deve permitir que o administrador registre eventuais sobras de comida.
    \item O sistema deve emitir alertas quando o estoque mínimo de um ingrediente for atingido.
    \item O sistema deve permitir que o administrador consulte o prazo de validade de cada ingrediente.
\end{enumerate}

\subsection{Requisitos de Nutrição e Qualidade}
\begin{enumerate}[label=\textbf{RF-NUT-\arabic*}, leftmargin=*, align=left]
    \item O sistema deve conter um processo de aprovação do controle de qualidade dos alimentos.
    \item O sistema deve exibir ao usuário o cardápio do dia.
    \item O sistema deve possibilitar ao usuário o acesso às informações nutricionais dos alimentos.
    \item O sistema deve permitir que funcionários do RU registrem casos de contaminação de alimentos.
    \item O sistema deve alertar os funcionários responsáveis pela gestão do RU em casos de contaminação dos alimentos.
    \item O sistema deve permitir que o usuário avalie a qualidade da refeição recebida.
    \item O sistema deve permitir que o administrador informe os valores nutricionais de cada alimento.
    \item O sistema deve gerar relatórios nutricionais consolidados (por exemplo: média de calorias e nutrientes ofertados por semana).
\end{enumerate}

\section{Requisitos Não-Funcionais}

\subsection{Requisitos Gerais}
\begin{enumerate}[label=\textbf{RNF-GEN-\arabic*}, leftmargin=*, align=left]
    \item O sistema deve seguir diretrizes de acessibilidade digital WCAG 2.1, garantindo que pessoas com deficiência possam utilizá-lo (ex.: contraste adequado, navegação por teclado, leitores de tela).
\end{enumerate}

\subsection{Requisitos de Fila}
\begin{enumerate}[label=\textbf{RNF-FIL-\arabic*}, leftmargin=*, align=left]
    \item O sistema deve atualizar a posição do usuário na fila em intervalos de no máximo 60 segundos.
    \item O sistema deve permanecer disponível durante todo o horário de funcionamento das refeições (mínimo de 2h15 a partir do início de cada turno), com disponibilidade mínima de 99\% no período crítico.
\end{enumerate}

\subsection{Requisitos de Estoque}
\begin{enumerate}[label=\textbf{RNF-EST-\arabic*}, leftmargin=*, align=left]
    \item O sistema deve atualizar a quantidade de ingredientes em estoque em tempo real, com base nas baixas de preparação registradas.
    \item O sistema deve ser capaz de suportar múltiplos acessos simultâneos (administradores, cozinha, nutricionista), garantindo consistência dos dados.
    \item O sistema deve proteger dados sensíveis (como custos e fornecedores) por meio de autenticação e controle de permissões baseado em papéis (RBAC).
\end{enumerate}

\subsection{Requisitos de Nutrição e Qualidade}
\begin{enumerate}[label=\textbf{RNF-NUT-\arabic*}, leftmargin=*, align=left]
    \item O sistema deve exibir o cardápio e as informações nutricionais em até 2 segundos após a solicitação.
    \item As avaliações de usuários sobre qualidade da refeição devem ser armazenadas de forma segura, protegidas contra acesso não autorizado.
\item As avaliações de usuários sobre qualidade da refeição devem:
\begin{itemize}
    \item ser registradas apenas por usuários autenticados;
    \item ser armazenadas de forma a garantir sua integridade (sem alteração não autorizada);
    \item ser acessíveis apenas a administradores e nutricionistas, não sendo exibidas publicamente com identificação pessoal do avaliador.
\end{itemize}

\end{enumerate}

\newpage
\section{Casos de Uso}

\subsection{Fila}

\subsubsection{UC-FILA-01 — Entrar na fila}
\textbf{Atores:} Usuário autenticado (via IdUFF).  

\textbf{Fluxo principal:}
\begin{enumerate}
    \item O usuário acessa o sistema IdUFF.
    \item O usuário abre a tela da fila e seleciona o RU desejado.
    \item O sistema exibe capacidade e tamanho da fila, bem como regras/políticas.
    \item O usuário (se elegível) escolhe a prioridade desejada.
    \item O usuário confirma a entrada na fila.
    \item O sistema valida a elegibilidade (RU aberto, critérios atendidos, etc.).
    \item O sistema registra o ingresso e atribui posição.
    \item O sistema calcula e apresenta a estimativa de espera (ETA).
    \item O sistema agenda as notificações conforme preferências do usuário.
\end{enumerate}

\textbf{Cenários alternativos:}
\begin{itemize}
    \item \textbf{2.a: RU fechado ou indisponível} — O sistema informa indisponibilidade e retorna ao passo 1.
    \item \textbf{4.a: Usuário não elegível à prioridade} — O sistema oculta opções de prioridade e retorna ao passo 4.
    \item \textbf{4.b: Validação manual de prioridade} — O usuário envia foto de comprovante; o sistema valida e retorna ao passo 4.
    \item \textbf{5.a: Saldo insuficiente} — O sistema recusa a entrada na fila e informa necessidade de regularização via IdUFF; após ajuste, retorna ao passo 5.
\end{itemize}

% ----------------------------------------------------------------
\subsubsection{UC-FILA-02 — Sair da fila (Desistência)}
\textbf{Atores:} Usuário autenticado.  

\textbf{Fluxo principal:}
\begin{enumerate}
    \item O usuário aciona a opção “Sair da fila”.
    \item O sistema solicita confirmação.
    \item O usuário confirma.
    \item O sistema remove o usuário e reordena a fila.
    \item O sistema atualiza ETA dos demais e confirma a saída.
    \item O sistema registra, opcionalmente, o motivo da desistência.
\end{enumerate}

\textbf{Cenários alternativos:}
\begin{itemize}
    \item \textbf{2.a: Usuário muito próximo de ser chamado} — O sistema informa política de desistência; o usuário confirma e segue ao passo 4 ou cancela e retorna ao passo 1.
\end{itemize}

% ----------------------------------------------------------------
\subsubsection{UC-FILA-03 — Consultar posição e ETA}
\textbf{Atores:} Usuário autenticado.  

\textbf{Fluxo principal:}
\begin{enumerate}
    \item O usuário abre a tela da fila no IdUFF.
    \item O sistema exibe posição atual, ETA e avisos do RU.
    \item O sistema atualiza automaticamente em intervalos definidos.
    \item O usuário pode solicitar atualização manual.
\end{enumerate}

\textbf{Cenários alternativos:}
\begin{itemize}
    \item \textbf{1.a: Usuário não está na fila} — O sistema oferece ação “Entrar na fila”; se aceito, executa UC-FILA-01.
    \item \textbf{2.a: Imprevistos no RU} — O sistema recalcula ETA, exibe aviso e retorna ao passo 2.
\end{itemize}

% ----------------------------------------------------------------
\subsubsection{UC-FILA-04 — Configurar notificações}
\textbf{Atores:} Usuário autenticado.  

\textbf{Fluxo principal:}
\begin{enumerate}
    \item O usuário abre a tela da fila pelo IdUFF.
    \item O usuário acessa “Preferências de notificação”.
    \item O usuário define limiares (ex.: faltam X pessoas/minutos).
    \item O sistema valida configurações e salva.
    \item O sistema confirma a ativação das preferências.
\end{enumerate}

% ----------------------------------------------------------------
\subsubsection{UC-FILA-05 — Confirmar presença}
\textbf{Atores:} Usuário autenticado.  

\textbf{Fluxo principal:}
\begin{enumerate}
    \item O sistema dispara um prompt de confirmação ao atingir um marco (ex.: top 50/top 20).
    \item O usuário confirma presença.
    \item O sistema mantém a posição e registra o timestamp.
\end{enumerate}

\textbf{Cenários alternativos:}
\begin{itemize}
    \item \textbf{2.a: Sem resposta no prazo} — O sistema remove o usuário da fila por inatividade e notifica a remoção.
    \item \textbf{2.b: Usuário escolhe desistir} — O sistema executa UC-FILA-02.
\end{itemize}

% ----------------------------------------------------------------
\subsubsection{UC-FILA-06 — Retroceder posições}
\textbf{Atores:} Usuário autenticado.  

\textbf{Fluxo principal:}
\begin{enumerate}
    \item O usuário abre a tela da fila pelo IdUFF.
    \item O usuário seleciona “Retroceder”.
    \item O sistema solicita a matrícula do usuário alvo.
    \item O usuário confirma a operação.
    \item O sistema altera a posição e reordena a fila.
    \item O sistema recalcula o ETA e confirma nova posição.
\end{enumerate}

\textbf{Cenários alternativos:}
\begin{itemize}
    \item \textbf{3.a: Matrícula inválida} — O sistema retorna erro e volta ao passo 3.
    \item \textbf{4.a: Limite excedido} — O sistema informa a regra e bloqueia a operação, retornando ao passo 2.
\end{itemize}

% ----------------------------------------------------------------
\subsubsection{UC-FILA-07 — Enviar feedback da experiência}
\textbf{Atores:} Usuário autenticado.  

\textbf{Fluxo principal:}
\begin{enumerate}
    \item O sistema oferece formulário curto após atendimento ou desistência.
    \item O usuário avalia tempo/organização/satisfação e insere comentários.
    \item O sistema registra o feedback e apresenta agradecimento.
\end{enumerate}

\textbf{Cenários alternativos:}
\begin{itemize}
    \item \textbf{1.a: Usuário ignora a solicitação} — O sistema não insiste e encerra o fluxo.
\end{itemize}

% ----------------------------------------------------------------
\subsubsection{UC-FILA-08 — Ver painel em tempo real (Admin)}
\textbf{Atores:} Administrador.  

\textbf{Fluxo principal:}
\begin{enumerate}
    \item O administrador acessa o painel.
    \item O sistema exibe pessoas na fila, ETA médio e horários de pico.
    \item O administrador aplica filtros (turno, campus, período).
\end{enumerate}

\textbf{Cenários alternativos:}
\begin{itemize}
    \item \textbf{2.a: Dados de catraca indisponíveis} — O sistema sinaliza degradação e utiliza dados estimados.
\end{itemize}

% ----------------------------------------------------------------
\subsubsection{UC-FILA-09 — Configurar capacidade e prioridades (Admin)}
\textbf{Atores:} Administrador.  

\textbf{Fluxo principal:}
\begin{enumerate}
    \item O administrador abre “Configurações de operação”.
    \item O sistema valida permissões de acesso.
    \item O administrador define a capacidade simultânea.
    \item O administrador configura critérios de prioridade e pesos.
    \item O administrador define abertura/fechamento do RU.
    \item O sistema valida consistência e salva.
    \item O sistema aplica as regras ao motor de fila e recalcula ETA.
\end{enumerate}

\textbf{Cenários alternativos:}
\begin{itemize}
    \item \textbf{2.a: Nível de acesso insuficiente} — O sistema bloqueia a operação e retorna à tela inicial.
\end{itemize}

% ----------------------------------------------------------------
\subsubsection{UC-FILA-10 — Conciliar eventos das catracas}
\textbf{Atores:} Administrador, Sistema de catracas.  

\textbf{Fluxo principal:}
\begin{enumerate}
    \item A catraca envia evento de entrada/saída.
    \item O sistema registra o evento e atualiza contagem.
    \item O sistema cruza eventos com a fila (chamados vs. entrados).
    \item O sistema sinaliza divergências e gera lista de conferência.
\end{enumerate}

\textbf{Cenários alternativos:}
\begin{itemize}
    \item \textbf{1.a: Falha de comunicação} — O sistema enfileira eventos e sinaliza degradação.
    \item \textbf{3.a: Divergência persistente} — O sistema gera ticket para verificação manual.
\end{itemize}

% ----------------------------------------------------------------
\subsubsection{UC-FILA-11 — Gerar relatórios (Admin)}
\textbf{Atores:} Administrador.  

\textbf{Fluxo principal:}
\begin{enumerate}
    \item O administrador solicita relatório.
    \item O administrador escolhe período e indicadores.
    \item O sistema consolida dados de fila, catracas e feedbacks.
    \item O sistema apresenta visualização e permite download.
\end{enumerate}

\textbf{Cenários alternativos:}
\begin{itemize}
    \item \textbf{3.a: Volume de dados elevado} — O sistema avisa sobre maior tempo de processamento e sugere filtros.
\end{itemize}

\subsection{Estoque}

\subsubsection{UC-ESTOQUE-01 — Preparo da comida}
\textbf{Atores:} Responsável pela retirada (RPR).  

\textbf{Visão geral:} Preparar alimentos para consumo, dando baixa em insumos e atualizando o estoque de refeições prontas.  

\textbf{Referência cruzada:} RF-EST-1, RF-EST-2, RF-EST-3, RF-EST-4, RF-EST-5, RF-EST-6, RF-EST-7, RF-EST-14.  

\textbf{Pré-condição:} Há insumos necessários em estoque.  

\textbf{Pós-condição:} Insumos são reduzidos e quantidade de comida pronta é atualizada.  

\textbf{Fluxo principal:}
\begin{enumerate}
    \item O responsável retira a quantidade necessária de insumos.
    \item O responsável dá baixa no estoque.
    \item O sistema chama UC-ESTOQUE-07 (Atualizar estoque de insumos).
    \item O sistema atualiza a quantidade de alimentos prontos.
    \item Os cozinheiros preparam a comida.
\end{enumerate}

\textbf{Cenários alternativos:}
\begin{itemize}
    \item \textbf{3.a: Limite mínimo identificado} — O sistema alerta (RF-EST-14); o responsável notifica compras. Retorna ao passo 4.
    \item \textbf{5.a: Erro no preparo} — A comida é descartada; fluxo retorna ao passo 1.
\end{itemize}

% -----------------------------------------------------------
\subsubsection{UC-ESTOQUE-02 — Envio da comida}
\textbf{Atores:} RU remetente e RU destinatário.  

\textbf{Visão geral:} O RU remetente envia refeições prontas a outro RU, com baixa no estoque e notificação.  

\textbf{Referência cruzada:} RF-EST-4, RF-EST-6, RF-EST-11.  

\textbf{Pré-condição:} Há comida pronta suficiente em estoque.  

\textbf{Pós-condição:} Estoque atualizado e RU destinatário notificado.  

\textbf{Fluxo principal:}
\begin{enumerate}
    \item O responsável consulta a quantidade solicitada.
    \item O responsável retira a quantidade do estoque.
    \item O responsável dá baixa no estoque.
    \item O sistema atualiza registros.
    \item O responsável registra as informações do pedido.
    \item O responsável confirma a saída da comida.
    \item O sistema notifica o RU destinatário.
    \item Chama UC-ESTOQUE-04 (Receber comida).
\end{enumerate}

% -----------------------------------------------------------
\subsubsection{UC-ESTOQUE-03 — Pedido de comida}
\textbf{Atores:} RU destinatário e RU fornecedor.  

\textbf{Visão geral:} Um RU solicita mais comida a outro.  

\textbf{Referência cruzada:} RF-EST-4, RF-EST-6, RF-EST-7, RF-EST-8, RF-EST-9, RF-EST-10.  

\textbf{Pré-condição:} RU destinatário precisa de comida adicional.  

\textbf{Pós-condição:} Pedido aceito ou recusado.  

\textbf{Fluxo principal:}
\begin{enumerate}
    \item O RU destinatário envia solicitação.
    \item O RU fornecedor recebe notificação.
    \item O RU fornecedor aceita o pedido.
    \item O sistema notifica o RU destinatário da aceitação.
    \item Chama UC-ESTOQUE-02 (Envio da comida).
\end{enumerate}

\textbf{Cenários alternativos:}
\begin{itemize}
    \item \textbf{3.a: Pedido negado} — O RU fornecedor recusa e notifica; o RU destinatário recebe negativa; fluxo encerrado.
\end{itemize}

% -----------------------------------------------------------
\subsubsection{UC-ESTOQUE-04 — Receber comida}
\textbf{Atores:} RU fornecedor e RU destinatário.  

\textbf{Visão geral:} O RU destinatário recebe, confere e atualiza o estoque de refeições prontas.  

\textbf{Referência cruzada:} RF-EST-4, RF-EST-6, RF-EST-8, RF-EST-10.  

\textbf{Pré-condição:} Envio concluído e registrado.  

\textbf{Pós-condição:} Estoque do RU destinatário atualizado.  

\textbf{Fluxo principal:}
\begin{enumerate}
    \item O RU destinatário confere a comida recebida.
    \item O RU destinatário notifica recebimento.
    \item O RU fornecedor recebe confirmação.
    \item O RU destinatário atualiza estoque no sistema.
\end{enumerate}

\textbf{Cenários alternativos:}
\begin{itemize}
    \item \textbf{1.a: A comida não chegou} — O RU destinatário registra divergência; chama UC-ESTOQUE-03.
    \item \textbf{1.b: Quantidade insuficiente} — O RU destinatário registra divergência; pode solicitar o restante; chama UC-ESTOQUE-03.
\end{itemize}

% -----------------------------------------------------------
\subsubsection{UC-ESTOQUE-05 — Gerar relatório}
\textbf{Atores:} Supervisor.  

\textbf{Visão geral:} O supervisor solicita relatórios personalizados de consumo e estoque.  

\textbf{Referência cruzada:} RF-EST-12, RF-EST-15.  

\textbf{Pré-condição:} Há dados suficientes.  

\textbf{Pós-condição:} Relatório gerado e exibido.  

\textbf{Fluxo principal:}
\begin{enumerate}
    \item O supervisor escolhe parâmetros.
    \item O supervisor solicita relatório.
    \item O sistema consolida dados.
    \item O sistema exibe relatório e permite download.
\end{enumerate}

\textbf{Cenários alternativos:}
\begin{itemize}
    \item \textbf{3.a: Erro de consolidação} — O sistema notifica erro; fluxo encerrado.
\end{itemize}


% -----------------------------------------------------------
\subsubsection{UC-ESTOQUE-06 — Fim de serviço}
\textbf{Atores:} Supervisor.  

\textbf{Visão geral:} O supervisor atualiza o sistema com a quantidade de refeições que sobraram ao final do expediente.  

\textbf{Referência cruzada:} RF-EST-12, RF-EST-13.  

\textbf{Pré-condição:} O horário de funcionamento do RU chegou ao fim.  

\textbf{Pós-condição:} É registrado no sistema a informação sobre a quantidade de comida que sobrou.  

\textbf{Fluxo principal:}
\begin{enumerate}
    \item O responsável registra o encerramento do serviço no sistema.
    \item O responsável informa a quantidade de comida pronta que sobrou.
    \item O sistema atualiza os registros de estoque com os valores informados.
\end{enumerate}

% -----------------------------------------------------------
\subsubsection{UC-ESTOQUE-07 — Atualizar estoque de insumos}
\textbf{Atores:} Administrador ou responsável pelo estoque.  

\textbf{Visão geral:} O sistema atualiza o estoque de insumos em função de entradas ou saídas (uso em preparo ou recebimento de novos insumos).  

\textbf{Referência cruzada:} RF-EST-1, RF-EST-3, RF-EST-5.  

\textbf{Gatilho:} UC-ESTOQUE-01 (Preparo da comida) ou UC-ESTOQUE-08 (Receber insumos).  

\textbf{Pré-condição:} Há movimentação de insumos (entrada ou saída).  

\textbf{Pós-condição:} Quantidade final de insumos refletida corretamente no estoque.  

\textbf{Fluxo principal:}
\begin{enumerate}
    \item O sistema identifica movimentação (entrada ou saída de insumos).
    \item O sistema calcula a nova quantidade de insumos.
    \item O sistema atualiza o estoque com o valor corrigido.
\end{enumerate}

% -----------------------------------------------------------
\subsubsection{UC-ESTOQUE-08 — Receber insumos}
\textbf{Atores:} Responsável pelo recebimento (RU remetente).  

\textbf{Visão geral:} O RU recebe insumos, contabiliza as quantidades e atualiza o sistema.  

\textbf{Referência cruzada:} RF-EST-1, RF-EST-3, RF-EST-5.  

\textbf{Pré-condição:} Há novos insumos entregues ao RU.  

\textbf{Pós-condição:} A quantidade final de insumos em estoque é incrementada corretamente.  

\textbf{Fluxo principal:}
\begin{enumerate}
    \item Os insumos chegam ao RU.
    \item O responsável contabiliza a quantidade recebida.
    \item O sistema chama UC-ESTOQUE-07 (Atualizar estoque de insumos).
\end{enumerate}

\subsection{Nutrição e Qualidade}

\subsubsection{UC-NUTRI-01 — Aprovação do controle de qualidade}
\textbf{Atores:} Administrador.  

\textbf{Referência cruzada:} RF-NUT-1.  

\textbf{Pré-condição:} Administrador logado no sistema.  

\textbf{Fluxo principal:}
\begin{enumerate}
    \item A equipe de controle de qualidade vai ao local de preparo.
    \item A equipe averigua o estado da comida.
    \item O administrador aprova a qualidade da comida no sistema.
    \item O sistema registra a aprovação.
    \item O sistema exibe ao administrador o status da aprovação.
\end{enumerate}

\textbf{Cenários alternativos:}
\begin{itemize}
    \item \textbf{2.a: A comida não passa pelo controle de qualidade}  
    \begin{enumerate}
        \item O administrador reprova a comida no sistema.
        \item O sistema registra a reprovação.
        \item O sistema notifica supervisores responsáveis sobre o incidente.
        \item O sistema exibe ao administrador o status da reprovação.
        \item Fim do caso de uso.
    \end{enumerate}
\end{itemize}

% -----------------------------------------------------------
\subsubsection{UC-NUTRI-02 — Consulta do cardápio}
\textbf{Atores:} Cliente.  

\textbf{Referência cruzada:} RF-NUT-2, RF-NUT-3.  

\textbf{Fluxo principal:}
\begin{enumerate}
    \item O cliente acessa a área de cardápio do dia.
    \item O sistema lista todos os restaurantes abertos.
    \item O cliente seleciona um restaurante.
    \item O sistema exibe o cardápio do dia incluindo informações nutricionais.
\end{enumerate}

\textbf{Cenários alternativos:}
\begin{itemize}
    \item \textbf{1.a: Cliente não está conectado} — Caso de uso incluído: Login.
\end{itemize}

% -----------------------------------------------------------
\subsubsection{UC-NUTRI-03 — Relatar contaminação}
\textbf{Atores:} Funcionário.  

\textbf{Referência cruzada:} RF-NUT-4, RF-NUT-5.  

\textbf{Pré-condição:} Funcionário logado no sistema; caso de contaminação identificado.  

\textbf{Fluxo principal:}
\begin{enumerate}
    \item O funcionário acessa a área “Relatos de contaminação”.
    \item O sistema exibe formulário.
    \item O funcionário descreve o ocorrido.
    \item O funcionário envia o relato.
    \item O sistema registra o relato.
    \item O sistema notifica supervisores.
\end{enumerate}

% -----------------------------------------------------------
\subsubsection{UC-NUTRI-04 — Registrar valores nutricionais}
\textbf{Atores:} Administrador.  

\textbf{Referência cruzada:} RF-NUT-7.  

\textbf{Pré-condição:} Administrador logado no sistema.  

\textbf{Fluxo principal:}
\begin{enumerate}
    \item O administrador acessa a seção de alimentos.
    \item O sistema exibe os alimentos registrados.
    \item O administrador inicia edição de um alimento.
    \item O sistema exibe as informações do alimento.
    \item O administrador edita as informações nutricionais.
    \item O administrador finaliza alterações.
    \item O sistema salva as mudanças.
\end{enumerate}

% -----------------------------------------------------------
\subsubsection{UC-NUTRI-05 — Acessar relatórios nutricionais consolidados}
\textbf{Atores:} Administrador.  

\textbf{Referência cruzada:} RF-NUT-8.  

\textbf{Pré-condição:} Administrador logado no sistema.  

\textbf{Fluxo principal:}
\begin{enumerate}
    \item O administrador acessa a seção de relatórios nutricionais.
    \item O sistema gera relatório consolidado com base nos cardápios semanais.
    \item O sistema exibe o relatório para o administrador.
\end{enumerate}

% Adicione esta seção após os Casos de Uso e antes do \end{document}

\section{Diagramas}

\subsection{Diagrama Geral de Casos de Uso}

A Figura~\ref{fig:diagrama_geral} apresenta a visão geral do sistema, mostrando todos os atores e casos de uso principais dos três módulos integrados: Fila Virtual, Estoque e Nutrição/Qualidade.

\begin{figure}[htbp]
    \centering
    \includegraphics[width=0.9\textwidth]{diagramas/diagrama-geral.png}
    \caption{Diagrama geral de casos de uso do Sistema de Gestão dos RUs da UFF}
    \label{fig:diagrama_geral}
\end{figure}

O diagrama ilustra as interações entre os diferentes perfis de usuários (Usuário Final, Administrador, Funcionário, Nutricionista) e as principais funcionalidades do sistema, demonstrando a integração entre os módulos de gestão.

\subsection{Diagramas por Módulo}

\subsubsection{Módulo Fila Virtual}
A Figura~\ref{fig:diagrama_fila} detalha os casos de uso específicos do módulo de fila virtual.

\begin{figure}[htbp]
    \centering
    \includegraphics[width=0.8\textwidth]{diagramas/diagrama-fila.png}
    \caption{Diagrama de casos de uso - Módulo Fila Virtual}
    \label{fig:diagrama_fila}
\end{figure}

\subsubsection{Módulo Estoque}
A Figura~\ref{fig:diagrama_estoque} apresenta os casos de uso do módulo de gestão de estoque.

\begin{figure}[htbp]
    \centering
    \includegraphics[width=0.8\textwidth]{diagramas/diagrama-estoque.png}
    \caption{Diagrama de casos de uso - Módulo Estoque}
    \label{fig:diagrama_estoque}
\end{figure}

\subsubsection{Módulo Nutrição e Qualidade}
A Figura~\ref{fig:diagrama_nutricao} ilustra os casos de uso relacionados à gestão nutricional.

\begin{figure}[htbp]
    \centering
    \includegraphics[width=0.8\textwidth]{diagramas/diagrama-nutricao.png}
    \caption{Diagrama de casos de uso - Módulo Nutrição e Qualidade}
    \label{fig:diagrama_nutricao}
\end{figure}

\newpage
\section{A melhorar}


Durante o desenvolvimento deste projeto, foi apontada pelo professor uma questão importante relacionada à distinção entre diagramas de casos de uso e fluxogramas de processo. 

Os relacionamentos de \textbf{inclusão} (\texttt{<<include>>}) nos diagramas de casos de uso \textbf{não representam sequência temporal} ou fluxo de execução, como seria o caso em um fluxograma. O relacionamento de inclusão indica que um caso de uso base sempre incorpora o comportamento de um caso de uso incluído, mas isso não define uma ordem específica de execução e foi nesse ponto que erramos ao construir os casos de uso para o módulo de estoque. 

É possível que ainda existam erros dessa natureza no trabalho, uma revisão conceitual seguida da revisão deste trabalho se faz necessária.

\end{document}
